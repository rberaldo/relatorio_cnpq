\chapter{Considerações finais}

%% Do relatório do Filipe
% As tarefas previstas para o período de setembro de 2009 a dezembro de 2011
% envolveram a análise, glosagem e indexação de 1396 synsets de substantivos,
% sendo 646 synsets do tipo semântico PROCESS e 750 synsets do tipo semântico
% ARTIFACT. Além da construção dessa parte da WN.Br, o bolsista realizou a
% divulgação do trabalho por meio da participação em seis eventos científicos,
% dos quais renderem cinco publicações.
% 
% As atividades até então desenvolvidas se deram em contato direto com a equipe
% de pesquisa coordenada pelo orientador, e iniciação do bolsista em pesquisas
% teórico-empíricas sobre Semântica Lexical, com ênfase na aplicação desta no
% âmbito da Linguística Computacional que colaboraram para a qualificação das
% atividades metodológicas realizadas.
% 
% Cumpre, por fim, informar que, com o término da graduação do bolsista Filipe
% de Freitas Araujo e na ausência de um bolsista substituto, a bolsa teve de
% ser cancelada, restando a análise e alinhamento de outros 250 synsets do
% domínio ARTIFACT, previstos para o período de janeiro de 2012 a julho de
% 2012.

%% Minha versão
As tarefas previstas para o período de~março de~2011 a~julho de~2012 envolveram
a análise, glosagem e indexação de 700~synsets do tipo~\textsc{artifact}. Além
da construção dessa parte da~\wnbr, o bolsista realizou a divulgação do
trabalho por meio da apresentação de cartazes em eventos científicos. Também
foram feitos esforços para permitir um melhor gerenciamento das várias partes
do projeto, especialmente da base de synsets.

As atividades até então desenvolvidas se deram em contato direto com a equipe
de pesquisa coordenada pelo orientador, e iniciação do bolsista em pesquisas
teórico-empíricas sobre Semântica Lexical, com ênfase na aplicação desta no
âmbito da Linguística Computacional que colaboraram para a qualificação das
atividades metodológicas realizadas.

Cumpre, por fim, informar que, este relatório marca o término da bolsa. O
projeto de construção da~\wnbr, entretanto, terá continuidade em outros
termos, entrando em nova fase, com a utilização do~GitHub para o gerenciamento
do projeto e relacionamento com outros pesquisadores e interessados no
  desenvolvimento e uso da~\wnbr.
