\chapter{Resultados}

% Original adaptado do Filipe
% Ao fim de um ano e cinco meses de trabalho, verifica-se o cumprimento de todos
% os objetivos programados para esse período, incluindo o aprofundamento dos
% conhecimento teóricos e metodológicos, concentrados em textos selecionados
% de~\cite{cruse}, \cite{fellbaum}, \cite{marrafa}, \cite{vossen},
% \cite{vossenetal} e~\cite{bento}; a realização do trabalho aplicado de
% alinhamento entre substantivos das duas bases, proporcionando ao bolsista a
% oportunidade de adquirir experiências em desenvolver pesquisa teórico-empírica,
% e também o conhecimento relacionado às wordnets norte-americana e europeia ---
% a EuroWordNet, uma rede semântica que inter-relaciona as wordnets das línguas
% da União Européia.

% Minha versão para comportar as tabelas
Ao fim de um ano e cinco meses de trabalho, verifica-se o cumprimento de todos
os objetivos programados para esse período, incluindo o aprofundamento dos
conhecimento teóricos e metodológicos, a realização do trabalho aplicado de
alinhamento entre substantivos das duas bases~(WN.Br e WN.Pr), o que
proporcionou ao bolsista a oportunidade de adquirir experiências em desenvolver
pesquisa teórico-empírica, e também os conhecimentos necessários para o
desenvolvimento de wordnets.
\enlargethispage{\baselineskip}

O quadro~\ref{panorama0} resume os objetivos para os primeiros cinco~meses de
trabalho; o quadro~\ref{panorama1} apresenta os objetivos para os doze meses
finais. Os Apêndices~A e~B apresentam, sob a forma de tabelas, os resultados
obtidos durante o período da bolsa: todos os synsets e glosas construídos para
o português, com os seus respectivos alinhamentos aos synsets do inglês.

% Tabela dos primeiros cinco meses
\begin{table}[!h]\footnotesize
  \centering
  \begin{tabularx}{\linewidth}{ l X X } 
    \toprule
    \textbf{Ao final de} & \textbf{Realizar as atividades} & \textbf{Com a previsão dos seguintes resultados} \\
    \midrule
    5 meses & Além dos estudos teórico-metodológicos, as atividades incluem,
sobretudo, a especificação das frases-exemplos para cada um dos itens lexicais
constitutivos de cada synset do português que puder ser indexado a cada um dos
200 synsets do inglês~(\textsc{artifact}), a análise, a revisão e a glosagem
dos synsets de substantivos do português e o seu alinhamento ao respectivo
synset do inglês (a indexação). & Domínio das técnicas de análise
léxico-semântica, inserção dos synsets indexados na base da WordNet.Br,
estabelecendo da correspondência entre estes e os seus equivalentes semânticos
na WordNet de Princeton e a especificação das relação de antonímia, hiponímia e
meronímia entre os synsets do português, contribuindo assim para a construção
da base dos substantivos do domínio~\textsc{artifact} da WordNet.Br.~(00300 a
00499) \\
    \bottomrule
  \end{tabularx}
  \caption{Panorama dos primeiros cinco meses}
  \label{panorama0}
\end{table}

% Tabela dos últimos doze meses
\begin{table}[!h]\footnotesize
  \centering
  \begin{tabularx}{\linewidth}{ l X X } 
    \toprule
    \textbf{Ao final de} & \textbf{Realizar as atividades} & \textbf{Com a previsão dos seguintes resultados} \\
    \midrule
    12 meses & Além dos estudos teórico-metodológicos, as atividades incluem,
sobretudo, a especificação das frases-exemplos para cada um dos itens lexicais
constitutivos de cada synset do português que puder ser indexado a cada um dos
200 synsets do inglês~(\textsc{artifact}), a análise, a revisão e a glosagem
dos synsets de substantivos do português e o seu alinhamento ao respectivo
synset do inglês (a indexação). & Domínio das técnicas de análise
léxico-semântica, inserção dos synsets indexados na base da WordNet.Br,
estabelecendo da correspondência entre estes e os seus equivalentes semânticos
na WordNet de Princeton e a especificação das relação de antonímia, hiponímia e
meronímia entre os synsets do português, contribuindo assim para a construção
da base dos substantivos do domínio~\textsc{artifact} da WordNet.Br.~(01000 a
01499) \\
    \bottomrule
  \end{tabularx}
  \caption{Panorama dos doze meses finais}
  \label{panorama1}
\end{table}

%\begin{table}[h]\footnotesize
%\tablefirsthead{\toprule \textbf{\ili/número na base} & \textbf{Relação semântica} & \textbf{WordNet.Br} & \textbf{WordNet.Pr} \\ \midrule}
%\tablehead{\toprule \textbf{\ili/número na base} & \textbf{Relação semântica} & \textbf{WordNet.Br} & \textbf{WordNet.Pr} \\}
%\tabletail{\bottomrule}
%  \begin{xtabular}[h]{m{.15\textwidth} m{.25\textwidth} m{.22\textwidth} m{.22\textwidth}}
%    02626354 & \eqsyn & \synset{antifúngico, agente antifúngico, fungicida, antimicótico, agente antimicótico} & \synset{\en{antifungal, antifungal agent, fungicide, antimycotic, antimycotic agent}} \\
%    \hline
%    02626693 & \eqsyn & \synset{traje anti-G} & \synset{\en{anti-G suit, G suit}} \\
%    \hline
%    02626842 & \eqsyn & \synset{anti-histamínico} & \synset{\en{antihistamine}} \\
%    \hline
%    02627277 & \eqsyn & \synset{antihipertensivo, droga antihipertensiva} & \synset{\en{antihypertensive, antihypertensive drug}} \\
%    \hline
%    02627637 & \eqsyn & \synset{anti-inflamatório, droga anti-inflamatória} & \synset{\en{anti-inflammatory, anti-inflammatory drug}} \\
%    \hline
%    02627912 & \eqhypo & \synset{capa protetora} & \synset{\en{antimacassar}} \\
%    \hline
%    02628045 & \eqsyn & \synset{antimalárico, droga antimalárica} & \synset{\en{antimalarial, antimalarial drug}} \\
%    \hline
%    02628263 & \eqsyn & \synset{antimetabólito} & \synset{\en{antimetabolite}} \\
%    \hline
%    02628446 & \eqsyn & \synset{antimicina} & \synset{\en{antimycin}} \\
%    \hline
%    02628555 & \eqsyn & \synset{antineoplásico, droga antineoplásica, medicamento anticancerígeno} & \synset{\en{antineoplasic, antineoplasic drug, cancer drug}} \\
%    \hline
%    02629076 & \eqsyn & \synset{antibiótico antineoplásico} & \synset{\en{antineoplasic antibiotic}} \\
%    \bottomrule
%  \end{xtabular}
%  \caption{Síntese dos primeiros cinco meses}
%  \label{synsets0}
%\end{table}
%
%%%%%%%%%%%%%%% SEGUNDA PARTE %%%%%%%%%%%%%%%
%
%\begin{table}[h]\footnotesize
%\tablefirsthead{\toprule \textbf{\ili/número na base} & \textbf{Relação semântica} & \textbf{WordNet.Br} & \textbf{WordNet.Pr} \\ \midrule}
%\tablehead{\toprule \textbf{\ili/número na base} & \textbf{Relação semântica} & \textbf{WordNet.Br} & \textbf{WordNet.Pr} \\}
%\tabletail{\bottomrule}
%  \begin{xtabular}[h]{m{.15\textwidth} m{.25\textwidth} m{.22\textwidth} m{.22\textwidth}}
%    02741864 & \eqsyn & \synset{laboratório de biologia} & \synset{\en{biology lab, biology laboratory, bio lab}} \\
%    \hline
%    02741989 & \eqsyn & \synset{bioscópio} & \synset{\en{bioscope}} \\
%    \hline
%    02742159 & \eqsyn & \synset{arma biológica} & \synset{\en{bioweapon, biological weapon, bioarm}} \\
%    \hline
%    02742429 & \eqsyn & \synset{biplano} & \synset{\en{biplane}} \\
%    \hline
%    02742540 & \eqsyn & \synset{biprisma} & \synset{\en{biprism}} \\
%    \hline
%    02742801 & \eqsyn & \synset{canoa de casca de bétula} & \synset{\en{birchbark canoe, birchbark, birch bark}} \\
%    \hline
%    02742930 & \eqsyn & \synset{banho de pássaro, banho de ave} & \synset{\en{birdbath}} \\
%    \hline
%    02743048 & \eqsyn & \synset{gaiola} & \synset{\en{birdcage}} \\
%    \hline
%    02743321 & \eqsyn & \synset{casa de passarinho} & \synset{\en{birdhouse}} \\
%    \hline
%    02743414 & \eqsyn & \synset{munição para caça} & \synset{\en{bird shot, buckshot, duck shot}} \\
%    \hline
%    02743546 & \eqsyn & \synset{barrete} & \synset{\en{biretta}} \\
%    \hline
%    02743671 & \eqsyn & \synset{bispo} & \synset{\en{bishop}} \\
%    \hline
%    02743829 & \eqsyn & \synset{bistrô} & \synset{\en{bistro}} \\
%    \hline
%    02743922 & \eqsyn & \synset{broca} & \synset{\en{bit}} \\
%    \hline
%    02744329 & \eqsyn & \synset{freio} & \synset{\en{bit}} \\
%    \hline
%    02744617 & \eqsyn & \synset{placa de mordida} & \synset{\en{bite plate, biteplate}} \\
%    \hline
%    02745628 & \eqsyn & \synset{preto} & \synset{\en{black}} \\
%    \hline
%    02745747 & \eqsyn & \synset{preta} & \synset{\en{black }} \\
%    \hline
%    06267282 & \eqsyn & \synset{preto-e-branco} & \synset{\en{print, black and white}} \\
%    \hline
%    02745998 & \eqsyn & \synset{lousa, quadro, quadro-negro} & \synset{\en{blackboard, chalkboard}} \\
%    \bottomrule
%  \end{xtabular}
%  \caption{Síntese dos doze meses finais}
%  \label{synsets1}
%\end{table}
