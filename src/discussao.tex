\pagebreak
\chapter{Discussão}

% Introdução à terminologia
Na busca de um synset do português que seja conceitualmente equivalente ao do
inglês, nem sempre ocorre um alinhamento direto, devido às especificidades de
cada uma das línguas e, sobretudo, às lacunas lexicais que são constatadas
durante o processo de análise.

%\enlargethispage{\baselineskip}
Segundo~\citeonline{peters}, há seis tipos de relações interlinguais. O
alinhamento direto é chamado~\eqsyn. As outras relações mais importantes
são~\cite[p.~225, tradução livre]{peters}:

	\begin{itemize}
        \item \eqnsyn\ quando um sentido corresponde a vários \ili s
          simultaneamente;
        \item \eqhyper\ quando um sentido é mais específico que qualquer \ili\
          disponível;
        \item \eqhypo\ quando um sentido pode ser ligado apenas a \ili s mais
          específicos.
	\end{itemize}

% Exemplos de alinhamento direto
Exemplo de \eqsyn\ é o synset da \wnpr\ \synset{\en{bread knife}}, que se
alinha diretamente com o synset da \wnbr\ \synset{faca de pão}. Nesse caso,
verificamos que as duas línguas --- o inglês norte-americano e o português
brasileiro --- revestem lexicalmente o mesmo conceito de uma faca especial para
cortar pães. Mesmo que esse tipo de alinhamento pareça bastante simples,
existem algumas questões a serem levantadas. Algumas são de cunho linguístico e
se relacionam às especificidades de cada língua, como no caso do synset
\synset{\en{brush2}} da \wnpr. Nesse caso, o synset preenche lexicalmente um
``utensílio que tem pelos ou cerdas firmemente presas a um cabo''. Em português
brasileiro, existem ao menos dois possíveis synsets que se encaixam nessa
descrição: \synset{escova} e \synset{pincel}. A dificuldade deriva do fato que,
em português, o conceito é bipartido em função do uso das escovas e pincéis;
podemos pensar que escovas \emph{limpam} ao passo que pincéis \emph{espalham}
(tinta, claras de ovos etc.). No inglês, essa diferenciação ocorre nos termos
\textit{\en{cleaning brush}} e \textit{\en{paint brush}}, ambos \emph{tipos} de
\en{\textit{brushes}}.

Outro tipo de problema decorre de possíveis problemas na formação dos synsets
da \wnpr. Esse é o caso de \synset{\en{bitmap, electronic image}}. Aqui é
necessária a compreensão técnica do termo “\en{\textit{bitmap}}”.
Segundo~\cite{wiki_bitmap}:

\begin{quote}
  [\ldots]~\en{bitmap or pixmap is a type of memory organization or image file
    format used to store digital images. The term bitmap comes from the computer
    programming terminology, meaning just a map of bits, a spatially mapped array
  of bits.}
\end{quote}

O problema é, portanto, que \en{\textit{bitmap}} é um \emph{tipo} de imagem
digital. Acreditamos que, por esse motivo, \synset{\en{bitmap}} deveria ser um
hipônimo de \synset{\en{electronic image}}.

Finalmente, alguns synsets são, por natureza, impossíveis de serem alinhados
por~\eqsyn. É o caso dos synsets \synset{\en{britches}} (termo informal para
\en{\textit{breeches}}, “calção”), \synset{\en{boot2}} (termo britânico para
“porta-malas”) e \synset{\en{burthen}} (termo informal para
\en{\textit{burden}}, “fardo”). Esses synsets apresentam variações no registro
(formal, informal, norte-americano, britânico) do inglês que não dizem respeito
ao conceito que preenchem. Para efeito de comparação, embora o termo “Sampa”
seja informal para “São Paulo”, ambas as palavras preenchem lexicalmente o
mesmo conceito --- o da capital do Estado de São Paulo --- e devem, por esse
motivo, formar um mesmo synset.

% Explicação do alinhamento por hiperonímia
O alinhamento por~\eqnsyn\ se dá quando um sentido pode se alinhar a múltiplos
sentidos na WN.Pr. Já o alinhamento por~\eqhyper\ acontece quando um conceito
falta no inglês ou na WN.Pr, como ilustrado na figura~\ref{caipirosca}.

	\tikzstyle{normal}=[circle,text=white,fill=black]
	\tikzstyle{superset}=[normal,inner sep=1em]

	\begin{figure}[!h]
	\centering
	\begin{tikzpicture}[scale=2]
	    \matrix [column sep=3em,row sep=1em]
	    {
	    \node (a) [normal] {WN.Br}; & \node (b) [superset] {WN.Pr}; \\
	    \node (a caption) {\parbox{2cm}{\centering\textbf{\{caipirosca\}}\\\small\textit{``tipo de capirinha, onde vodka é utilizada ao invés de cachaça''}}}; & \node (b caption) {\parbox{2cm}{\centering\textbf{\{vodka\}}\\\small\textit{``unaged colorless liquor originating in Russia''}}};\\
	    };
	    \begin{pgfonlayer}{background}
		\fill (a.center) -- (b.north east) -- (b.south east) -- (a.center);
	    \end{pgfonlayer}
	\end{tikzpicture}
	\caption{Relação do tipo \eqhyper. Adaptado de~\citeonline{debora}.}
	\label{caipirosca}
	\end{figure}

% Explicação do alinhamento por hiponímia
Quando a wordnet de Princeton apresenta um synset mais específico do que o
presente na wordnet a ser alinhada, acontece o alinhamento por~\eqhypo. O
exemplo dado por~\citeonline{debora} é representado na figura~\ref{dedo}.

	\begin{figure}[h]
	\centering
	\begin{tikzpicture}[scale=2]
	    \matrix [column sep=3em,row sep=1em]
	    {
	    & \node (b caption) {\parbox{3cm}{\centering\textbf{\{finger\}}\\\small\textit{``any of the terminal members of the hand (sometimes excepting the thumb)''}}};  \\
	    \node (a) [superset] {WN.Br}; & \node (b) [normal] {WN.Pr}; \\
	    \node (a caption) {\parbox{3cm}{\centering\textbf{\{dedo\}}\\\small\textit{``cada um dos cinco prolongamentos articulados que terminam as mãos e os pés do homem''}}}; & \node (b caption) {\parbox{3cm}{\centering\textbf{\{toe\}}\\\small\textit{``one of the digits of the foot''}}};\\
	    };
	    \begin{pgfonlayer}{background}
		\fill (b.center) -- (a.north east) -- (a.south east) -- (b.center);
	    \end{pgfonlayer}
	\end{tikzpicture}
	\caption{Relação do tipo \eqhypo. Adaptado de~\citeonline{debora}.}
	\label{dedo}
	\end{figure}

Embora o alinhamento por~\eqsyn\ seja o mais abundante em termos absolutos, o
alinhamento por~\eqhypo\ ocorre em grande número. Dos 500 synsets analisados
nos últimos doze meses de trabalho, aproximadamente 150~deles ($\approx$~30\%)
apresentam esse alinhamento. Exemplo desse tipo de alinhamento é o synset
da~\wnbr\ \synset{bomba guiada por laser, bomba inteligente}, que se alinha por
\eqhypo\ ao synset da \wnpr\ \synset{\en{Bunker Buster, Guided Bomb Unit-28,
GBU-28}}, que representa um tipo específico de bomba guiada por laser. Também
\synset{arado} foi alinhado dessa maneira a~\synset{\en{bull tongue}}, um tipo
de arado pesado usado em plantações de algodão.
