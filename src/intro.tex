\chapter{Introdução}

Este \textbf{Relatório Final} refere-se às atividades desenvolvidas pelo
bolsista Rafael Luis Beraldo no período de 1\textordmasculine~de março de~2011
a 31~de julho de~2012, após substituição da bolsista Débora Domiciano Garcia. O
Plano de Atividades em que este relatório se pauta sustenta-se no projeto do
orientador, \textit{``O Desenvolvimento da Base de Substantivos da WordNet.Br e
a sua Co-indexação com a WordNet de Princeton''}, que consiste na montagem de
parte da rede WordNet.Br (WN.Br) --- uma base relacional de unidades e
expressões lexicais do português do Brasil, construída a partir da investigação
de questões linguísticas~\cite{bentoetal,bento}, motivada pelo estudo da rede
WordNet de Princeton (WN.Pr), na versão \textit{WordNet 2.0}, e da metodologia
empregada em sua construção por pesquisadores do Laboratório de Ciências
Cognitivas da Universidade de Princeton, Estados
Unidos~\cite{miller,fellbaum}.\footnote{O termo \textit{wordnet} é
internacionalmente aceito para designar um tipo particular de base de
conhecimento lexical em que as relações que tecem a rede são relações
léxico-conceituais~\cite{marrafa}.} 

A WordNet.Br, como as demais wordnets, deverá constituir um recurso lexical
que, além de servir de instrumento complementar para o estudo de línguas,
poderá também ser utilizado em aplicações no âmbito do Processamento Automático
de Línguas Naturais (\textsc{pln}) e da Engenharia da Linguagem como, por
exemplo, sistemas de tradução automática, motores de busca na Internet (Google,
AltaVista, Lycos, entre outros), processadores e sumarizadores de texto,
aplicativos de recuperação de textos e de informação em bases de documentos e,
com a implementação do
\ili~(\foreignlanguage{english}{\textit{Inter-Lingual-Index}}), dicionários
digitais inglês--português/português--inglês para acesso gratuito online na
Internet.

Conforme os cronogramas de atividades (cf.~o~Plano de Atividades em~Anexo), de
março de~2011 a~julho de~2012, coube ao bolsista a realização de tarefas que
permitiram o aprimoramento dos seus conhecimentos sobre o funcionamento das
redes wordnet norte-americana e europeias, e a sua contribuição para o
desenvolvimento da rede wordnet brasileira.

As atividades de pesquisa deste projeto realizaram-se no \celic\ (Centro de
Estudos Linguísticos e Computacionais da Linguagem), localizado na Sala~29,
Prédio dos GPs, Faculdade de Ciências e Letras, \unesp, Campus de Araraquara,
sede do Grupo de Pesquisa ``Grupo de Estudos Linguístico-Computacionais da
Linguagem Humana'', liderado pelo orientador, cadastrado no \cnpq\ e
certificado pela Universidade.
